\documentclass[12pt,a4paper]{article}
\usepackage[utf8]{inputenc}
\usepackage[turkish]{babel}
\usepackage{amsmath,amsfonts,amssymb}
\usepackage{graphicx}
\usepackage{hyperref}
\usepackage{physics}
\usepackage{float}
\usepackage{fancyhdr}

\pagestyle{fancy}
\fancyhf{}
\lhead{İstanbul Üniversitesi Fizik Bölümü}
\rhead{FZKT2404 - Elektronik I - Analog Elektronik}
\lfoot{Ekrem Torun}
\rfoot{\thepage}

\begin{document}

\begin{titlepage}
    \centering
    \includegraphics[width=0.5\textwidth]{iu_logo.png}\\[1cm]
    {\scshape\LARGE İstanbul Üniversitesi\\}
    {\scshape\Large Fen Fakültesi - Fizik Bölümü\\[0.5cm]}
    \rule{\linewidth}{0.2mm} \\[0.4cm]
    { \huge \bfseries FZKT2404 - Elektronik I - Analog Elektronik\\
    Ders Notları\\[0.4cm] }
    \rule{\linewidth}{0.2mm} \\[1.5cm]
    
    {\large
    \begin{tabular}{rl}
        \textbf{Öğrenci:} & Ekrem Torun \\
        \textbf{Öğretim Elemanı:} & Doç. Dr. NESLİHAN ÜZAR KILIÇ \\
        \textbf{Akademik Dönem:} & 2024-2025 Bahar \\
    \end{tabular}}\\[2cm]
    
    {\large \today}
\end{titlepage}

\tableofcontents
\newpage

\section{Giriş}
Bu doküman, FZKT2404 Elektronik I - Analog Elektronik dersinin notlarını içermektedir.

\section{Temel Kavramlar}
% Ders notları buraya eklenecek

\section{Önemli Formüller}
% Formüller buraya eklenecek

\section{Örnek Problemler}
% Örnek problemler ve çözümleri buraya eklenecek

\end{document}
