\documentclass{article}
\usepackage{amsmath}
\usepackage{amssymb}

\begin{document}

\section{Modern Fizik Notları}

\subsection{Temel Veriler}
\begin{itemize}
    \item Elektron kütlesi: \( m_e = 9.109 \times 10^{-31} \) kg
    \item Proton kütlesi: \( m_p = 1.672 \times 10^{-27} \) kg
\end{itemize}

\subsection{Kütle Oranı ve Relativistik Hız}
Elektronun ve protonun durgun kütleleri \( m_{e0} \) ve \( m_{p0} \) olsun.
\[ m_{e0} = m_e \]
\[ m_{p0} = m_p \]

Kütlelerin oranı:
\[ \frac{m_{e0}}{m_{p0}} = \frac{1}{\lambda} \]

Buradan, relativistik kütle formülünü kullanarak:
\[ \frac{m_e}{m_p} = \frac{m_{e0}}{m_{p0}} \sqrt{\frac{1 - v^2/c^2}{1 - v^2/c^2}} \]
\[ \frac{m_{e0}}{m_{p0}} = \sqrt{1 - \frac{v^2}{c^2}} \]

Denklemi genişletip \( v \) için çözersek:
\[ \left( \frac{m_{e0}}{m_{p0}} \right)^2 = 1 - \frac{v^2}{c^2} \]
\[ \frac{v^2}{c^2} = 1 - \left( \frac{m_{e0}}{m_{p0}} \right)^2 \]
\[ v = c \sqrt{1 - \left( \frac{m_{e0}}{m_{p0}} \right)^2} \]

Değerleri yerine yazalım:
\( c = 2.998 \times 10^8 \) m/s (ışık hızı)
\[ \frac{m_{e0}}{m_{p0}} = \frac{9.109 \times 10^{-31}}{1.672 \times 10^{-27}} \approx 0.0005448 \]
\[ v = (2.998 \times 10^8) \sqrt{1 - (0.0005448)^2} \]
\[ v \approx 2.9979995 \times 10^8 \, \text{m/s} \]

\subsection{Sonuç}
İşlemi yaptığımızda sonuç \( v \approx 2.9979995 \times 10^8 \) m/s çıkıyor.

\end{document}
