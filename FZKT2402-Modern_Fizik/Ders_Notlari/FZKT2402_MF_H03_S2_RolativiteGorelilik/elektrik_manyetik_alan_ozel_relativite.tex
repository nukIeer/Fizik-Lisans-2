\documentclass[11pt,letterpaper,twocolumn]{fenbil}
\usepackage{amsmath}
\usepackage{amssymb}
\usepackage{graphicx}
\title{}
\author{}
\date{}

\begin{document}
\twocolumn[\begin{@twocolumnfalse}

\begin{minipage}{0.15\textwidth}{
    }
\end{minipage}
\hspace{25pt}
\begin{minipage}{0.75\textwidth}
\vspace{5mm}
\Large{\textbf{Perşembe Modern Fizik Dersi - Ders 1 (6 Mart 2025)\newline Elektrik ve Manyetik Alanlar: Özel Rölativistik Bakış
     }}
    \vspace{3mm}
    
    \large{\textbf{Hazırlayan}; Celal Ekrem Torun}
    \vspace{2mm}
    
    \fontsize{0.35cm}{0.5cm}\selectfont \textit{Fizik Bölümü, İstanbul Üniversitesi\newline 
    Beyazıt, Fatih, İstanbul, Türkiye}
    
\end{minipage}

\small

\end{@twocolumnfalse}]

\section{Giriş}
Bu ders notunda, elektrik ve manyetik alanların temel özelliklerini, kütle değişiminin özel rölativitedeki rolünü ve elektrik yükünün bu bağlamdaki korunumu prensibini inceleyeceğiz. Özel rölativite, uzay ve zamanın göreli doğasını anlamamızı sağlayan temel bir teoridir.

\section{Elektrik ve Manyetik Alanlar}
Elektrik ve manyetik alanlar, elektromanyetik kuvvetin iki temel bileşenidir. Elektrik alanlar, elektrik yükleri tarafından oluşturulur ve diğer yüklere kuvvet uygular. İlginç bir şekilde, manyetik alanlar elektrik alanın rölativistik bir etkisidir. Bu, hareketli bir yükün elektrik alanı, farklı bir referans çerçevesinden gözlemlendiğinde manyetik bir alan olarak algılanabileceği anlamına gelir.

Özel rölativite, elektrik ve manyetik alanların birbirine bağlı olduğunu gösterir. Bir referans çerçevesinde sadece elektrik alanı olarak görünen bir alan, başka bir referans çerçevesinde hem elektrik hem de manyetik alan olarak görülebilir. Bu dönüşüm, Lorentz dönüşümleri ile ifade edilir.

Faraday'ın indüksiyon yasası ve Ampère yasası, elektrik ve manyetik alanların birbirleriyle nasıl etkileşime girdiğini açıklar. Maxwell denklemleri, bu yasaları bir araya getirerek elektromanyetizmanın temelini oluşturur.

Elektrik alan \( \mathbf{E} \) ve manyetik alan \( \mathbf{B} \) arasındaki ilişki, Lorentz kuvveti ile ifade edilir:
\[ \mathbf{F} = q(\mathbf{E} + \mathbf{v} \times \mathbf{B}) \]
Burada \( q \) yükü, \( \mathbf{v} \) ise yükün hızıdır.

Maxwell denklemleri şunlardır:
\begin{align*}
\nabla \cdot \mathbf{E} &= \frac{\rho}{\varepsilon_0} \\
\nabla \cdot \mathbf{B} &= 0 \\
\nabla \times \mathbf{E} &= -\frac{\partial \mathbf{B}}{\partial t} \\
\nabla \times \mathbf{B} &= \mu_0 \mathbf{J} + \mu_0 \varepsilon_0 \frac{\partial \mathbf{E}}{\partial t}
\end{align*}
Burada \( \rho \) yük yoğunluğu, \( \mathbf{J} \) akım yoğunluğu, \( \varepsilon_0 \) boşluğun dielektrik katsayısı ve \( \mu_0 \) boşluğun manyetik geçirgenliğidir.

Bu denklemler, elektromanyetik dalgaların varlığını öngörür ve ışığın da bir elektromanyetik dalga olduğunu gösterir. Bu denklemler aynı zamanda, elektrik ve manyetik alanların birbirine göre nasıl değiştiğini de tanımlar.

\section{Kütle Değişimi Özel Rölativitede}
Özel rölativiteye göre, bir cismin kütlesi hıza bağlı olarak değişir. Durgun kütle \( m_0 \) olarak adlandırılan bir cismin kütlesi, \( v \) hızıyla hareket ederken \( m \) olarak ölçülür ve bu ilişki aşağıdaki Lorentz dönüşümü ile ifade edilir:
\[ m = \gamma m_0 = \frac{m_0}{\sqrt{1 - \frac{v^2}{c^2}}} \]
Burada \( c \) ışık hızıdır ve \( \gamma \) Lorentz faktörüdür. Bu denklem, bir cismin hızı arttıkça kütlesinin de arttığını gösterir. Ancak, bu kütle artışı, cismin eylemsizliğinin artması şeklinde yorumlanmalıdır.

Enerji ve kütle arasındaki ilişki ise Einstein'ın ünlü denklemi ile ifade edilir:
\[ E = mc^2 \]
Bu denklem, kütlenin enerjiye ve enerjinin kütleye dönüştürülebileceğini gösterir.

\section{Yükün Korunumu}
Özel rölativitede elektrik yükü değişmez bir niceliktir. Yani, bir parçacığın elektrik yükü, parçacığın hızından bağımsızdır. Bu, yükün korunumu yasasının rölativistik bir genellemesidir.

Yükün korunumu, dört-akım yoğunluğu \( J^\mu \) ile ifade edilir:
\[ \partial_\mu J^\mu = 0 \]
Burada \( J^\mu = (c\rho, \mathbf{J}) \) olup, \( \rho \) yük yoğunluğu ve \( \mathbf{J} \) akım yoğunluğudur. Bu denklem, yükün korunumu yasasının matematiksel ifadesidir ve yükün herhangi bir noktada kaybolmadığını veya oluşmadığını gösterir.

Elektrik yükünün korunumu, fiziksel süreçlerde toplam elektrik yükünün zamanla değişmediği anlamına gelir. Bu prensip, parçacık fiziği ve elektromanyetizma gibi birçok alanda temel bir öneme sahiptir.

\section{Genel Görelilik}
Genel görelilik, kütleçekimini uzay-zamanın eğriliği olarak tanımlayan bir teoridir. Einstein'ın genel görelilik teorisi, kütleçekimini sadece bir kuvvet olarak değil, uzay ve zamanın geometrik bir özelliği olarak ele alır. Bu teoriye göre, büyük kütleler uzay-zamanı büker ve bu bükülme, diğer cisimlerin hareketini etkiler.

\subsection{Gezegenler Arası Trigonometrik Etkiler}
Genel görelilik, ışığın büyük kütlelerin yakınından geçerken sapacağını öngörür. Bu sapma, yıldızların ve galaksilerin konumlarının gözlemlenmesinde trigonometrik etkilere yol açar. Örneğin, uzak bir galaksinin ışığı, önündeki bir başka galaksinin kütleçekimi tarafından bükülebilir, bu da galaksinin görüntüsünün bozulmasına veya çoğalmasına neden olabilir.

\subsection{Asansör ve Kütle Örnekleri}
Einstein'ın eşdeğerlik ilkesi, kütleçekimsel kuvvet ile ivmeli bir referans çerçevesindeki kuvvetin ayırt edilemez olduğunu belirtir. Bir asansör içinde bulunan bir gözlemci, asansör yukarı doğru ivmelenirken hissettiği kuvveti, kütleçekimsel bir kuvvet olarak yorumlayabilir. Benzer şekilde, serbest düşen bir asansördeki gözlemci, ağırlıksızlık hisseder çünkü kütleçekimi ve ivme birbirini dengeler.

\subsection{İvme Örnekleri}
Genel görelilik, ivmeli hareketin uzay-zamanı nasıl etkilediğini de açıklar. Örneğin, dönen bir disk üzerindeki bir gözlemci, merkezden uzaklaştıkça artan bir eylemsizlik kuvveti hisseder. Bu kuvvet, kütleçekimsel bir alan gibi davranır ve zamanın akışını etkiler. Dönen diskin kenarındaki bir saat, merkezdeki bir saate göre daha yavaş akar.

\section{Örnek Soru ve Çözümü}
Hareketli bir elektron, durgun bir elektronla çarpıştırılıyor. Sonuçta, bu elektronlara ek olarak bir elektron ve bir pozitron çifti çıkıyor. Pozitronun ne olduğunu anlatınız ve çarpışmadan sonra dört parçacık da aynı hıza sahipse, bu süreç için gerekli minimum kinetik enerjinin \( 6m_0c^2 \) olduğunu gösteriniz.

\subsection{Pozitron Nedir?}
Pozitron, elektronun анти parçacığıdır. Aynı kütleye ve spine sahip olup, elektrik yükü elektronun zıttıdır (yani +e).

\subsection{Çözüm}
Çarpışma öncesinde, hareketli elektronun enerjisi \( E_1 \) ve momentumu \( p_1 \), durgun elektronun enerjisi \( E_0 = m_0c^2 \) ve momentumu \( p_0 = 0 \) olsun. Çarpışma sonrasında, dört parçacığın her birinin enerjisi \( E_s \) ve momentumu \( p_s \) olsun.

Enerji korunumu:
\[ E_1 + E_0 = 4E_s \]
Momentum korunumu:
\[ p_1 + p_0 = 4p_s \]
\[ p_1 = 4p_s \]

Relativistik enerji-momentum ilişkisi:
\[ E^2 = (pc)^2 + (m_0c^2)^2 \]
Çarpışma öncesi hareketli elektron için:
\[ E_1^2 = (p_1c)^2 + (m_0c^2)^2 \]
Çarpışma sonrası her bir parçacık için:
\[ E_s^2 = (p_sc)^2 + (m_0c^2)^2 \]

\( p_1 = 4p_s \) olduğundan, \( p_s = \frac{p_1}{4} \) olur.
\[ E_s^2 = \left( \frac{p_1c}{4} \right)^2 + (m_0c^2)^2 \]
\[ E_s = \sqrt{\left( \frac{p_1c}{4} \right)^2 + (m_0c^2)^2} \]

Enerji korunumu denkleminden:
\[ E_1 + m_0c^2 = 4E_s \]
\[ E_1 = 4 \sqrt{\left( \frac{p_1c}{4} \right)^2 + (m_0c^2)^2} - m_0c^2 \]

\( E_1^2 = (p_1c)^2 + (m_0c^2)^2 \) olduğundan:
\[ \left( 4 \sqrt{\left( \frac{p_1c}{4} \right)^2 + (m_0c^2)^2} - m_0c^2 \right)^2 = (p_1c)^2 + (m_0c^2)^2 \]
Bu denklemi çözmek için, \( x = p_1c \) ve \( y = m_0c^2 \) diyelim:
\[ \left( 4 \sqrt{\left( \frac{x}{4} \right)^2 + y^2} - y \right)^2 = x^2 + y^2 \]
\[ 16 \left( \frac{x^2}{16} + y^2 \right) - 8y \sqrt{\frac{x^2}{16} + y^2} + y^2 = x^2 + y^2 \]
\[ x^2 + 16y^2 - 8y \sqrt{\frac{x^2}{16} + y^2} + y^2 = x^2 + y^2 \]
\[ 16y^2 = 8y \sqrt{\frac{x^2}{16} + y^2} \]
\[ 2y = \sqrt{\frac{x^2}{16} + y^2} \]
\[ 4y^2 = \frac{x^2}{16} + y^2 \]
\[ 3y^2 = \frac{x^2}{16} \]
\[ x^2 = 48y^2 \]
\[ x = \sqrt{48}y = 4\sqrt{3}y \]

Bu durumda, \( p_1c = 4\sqrt{3} m_0c^2 \) olur.
\[ E_1 = \sqrt{(p_1c)^2 + (m_0c^2)^2} = \sqrt{(4\sqrt{3} m_0c^2)^2 + (m_0c^2)^2} \]
\[ E_1 = \sqrt{48(m_0c^2)^2 + (m_0c^2)^2} = \sqrt{49(m_0c^2)^2} = 7m_0c^2 \]

Hareketli elektronun kinetik enerjisi:
\[ KE = E_1 - m_0c^2 = 7m_0c^2 - m_0c^2 = 6m_0c^2 \]
Bu nedenle, gerekli minimum kinetik enerji \( 6m_0c^2 \) dir.

\subsection{Gelecek Hafta: Dalgaların Parçacık Özelliği}
Gelecek hafta, dalgaların parçacık özelliği konusuna giriş yapacağız.

\section{Sonuç}
Elektrik ve manyetik alanlar, elektromanyetik etkileşimin temelini oluşturur. Özel rölativite, kütlenin hıza bağlı olarak değiştiğini gösterirken, elektrik yükünün korunumu prensibinin evrensel bir yasa olduğunu doğrular. Genel görelilik ise kütleçekimini uzay-zamanın eğriliği olarak tanımlar ve evrenin büyük ölçekli yapısını anlamamızda kritik bir rol oynar. Bu kavramlar, modern fiziğin temel taşlarıdır ve evrenin işleyişini anlamamızda kritik bir rol oynar.

\end{document}
