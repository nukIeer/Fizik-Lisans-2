\documentclass{fenbil}
\usepackage{amsmath}
\usepackage{amssymb}
\usepackage{physics}
\usepackage{graphicx}
\usepackage{geometry}
\usepackage{tikz}

\geometry{a4paper, margin=1in}

\title{Uygulama Dersi - Hafta 3}
\author{Koordinat Sistemleri ve Dönüşümleri}
\date{\today}

\begin{document}

\maketitle

\section{Soru 2: Silindirik Koordinat Sisteminin Ortogonalliği}

\subsection{Çözüm}

Silindirik koordinat sisteminde üç koordinat vardır: $\rho$ (radyal mesafe), $\theta$ (açısal koordinat) ve $z$ (yükseklik). Bu koordinat sisteminin ortogonal olduğunu göstermek için iki yöntem kullanabiliriz:

\subsubsection{1. Yöntem: Birim Vektörlerin Skaler Çarpımı}

Silindirik koordinat sistemindeki birim vektörler $\vec{e}_\rho$, $\vec{e}_\theta$ ve $\vec{e}_z$ olarak gösterilir. Bir koordinat sisteminin ortogonal olması için, bu birim vektörlerin birbirine dik olması gerekir. Matematiksel olarak:

\begin{align}
\vec{e}_\rho \cdot \vec{e}_\theta &= 0 \\
\vec{e}_\rho \cdot \vec{e}_z &= 0 \\
\vec{e}_\theta \cdot \vec{e}_z &= 0
\end{align}

Bu skaler çarpımların sıfır olması, vektörlerin birbirine dik olduğunu gösterir.

\subsubsection{2. Yöntem: Metrik Tensör Bileşenleri}

Alternatif olarak, koordinat eğrilerinin teğet vektörleri arasındaki skaler çarpımların sıfır olduğunu gösterebiliriz:

\begin{align}
\vec{t}_\rho \cdot \vec{t}_\theta &= 0 \\
\vec{t}_\rho \cdot \vec{t}_z &= 0 \\
\vec{t}_\theta \cdot \vec{t}_z &= 0
\end{align}

Bu iki yöntemden herhangi biri, silindirik koordinat sisteminin ortogonal olduğunu kanıtlar.

\section{Soru 3: Silindirik Koordinatlarda Birim Vektörler ve Dönüşüm Matrisi}

\subsection{Çözüm}

\subsubsection{Birim Vektörlerin Kartezyen Bileşenleri}

Silindirik koordinatlardaki birim vektörleri, kartezyen birim vektörler cinsinden ifade edelim:

\begin{align}
\vec{e}_\rho &= \cos\theta\, \vec{e}_x + \sin\theta\, \vec{e}_y \\
\vec{e}_\theta &= -\sin\theta\, \vec{e}_x + \cos\theta\, \vec{e}_y \\
\vec{e}_z &= \vec{e}_z
\end{align}

\subsubsection{Koordinat Dönüşümleri}

Silindirik koordinatlar ile kartezyen koordinatlar arasındaki ilişki:

\begin{align}
x &= \rho \cos\theta \\
y &= \rho \sin\theta \\
z &= z
\end{align}

\subsubsection{Dönüşüm Matrisi}

Silindirik birim vektörleri kartezyen birim vektörler cinsinden ifade eden dönüşüm matrisi:

\begin{align}
\begin{bmatrix} \vec{e}_\rho \\ \vec{e}_\theta \\ \vec{e}_z \end{bmatrix} = 
\begin{bmatrix} 
\cos\theta & \sin\theta & 0 \\
-\sin\theta & \cos\theta & 0 \\
0 & 0 & 1
\end{bmatrix}
\begin{bmatrix} \vec{e}_x \\ \vec{e}_y \\ \vec{e}_z \end{bmatrix}
\end{align}

Bu matrisin tersi, kartezyen birim vektörleri silindirik birim vektörler cinsinden ifade eder:

\begin{align}
\begin{bmatrix} \vec{e}_x \\ \vec{e}_y \\ \vec{e}_z \end{bmatrix} = 
\begin{bmatrix} 
\cos\theta & -\sin\theta & 0 \\
\sin\theta & \cos\theta & 0 \\
0 & 0 & 1
\end{bmatrix}
\begin{bmatrix} \vec{e}_\rho \\ \vec{e}_\theta \\ \vec{e}_z \end{bmatrix}
\end{align}

\subsubsection{Konum Vektörünün İfadesi}

Kartezyen koordinatlarda konum vektörü:
\begin{align}
\vec{r} = x\vec{e}_x + y\vec{e}_y + z\vec{e}_z
\end{align}

Silindirik koordinatlarda konum vektörünü ifade etmek için, kartezyen bileşenleri silindirik koordinatlar cinsinden yazıp, kartezyen birim vektörleri de silindirik birim vektörler cinsinden ifade edelim:

\begin{align}
\vec{r} &= \rho\cos\theta\vec{e}_x + \rho\sin\theta\vec{e}_y + z\vec{e}_z \\
&= \rho\cos\theta(\cos\theta\vec{e}_\rho - \sin\theta\vec{e}_\theta) + \rho\sin\theta(\sin\theta\vec{e}_\rho + \cos\theta\vec{e}_\theta) + z\vec{e}_z \\
&= \rho\cos^2\theta\vec{e}_\rho - \rho\cos\theta\sin\theta\vec{e}_\theta + \rho\sin^2\theta\vec{e}_\rho + \rho\sin\theta\cos\theta\vec{e}_\theta + z\vec{e}_z \\
&= \rho(\cos^2\theta + \sin^2\theta)\vec{e}_\rho + \rho(\sin\theta\cos\theta - \cos\theta\sin\theta)\vec{e}_\theta + z\vec{e}_z \\
&= \rho\vec{e}_\rho + z\vec{e}_z
\end{align}

Bu, silindirik koordinatlarda konum vektörünün standart ifadesidir.

\section{Skaler Alanların Dönüşümü}

Aşağıdaki skaler alanların silindirik koordinatlardaki ifadelerini bulalım:

\subsection{$u_1 = 2xy^2 + x^2$}

Kartezyen koordinatlardan silindirik koordinatlara dönüşüm formüllerini kullanarak:
\begin{align}
x &= \rho\cos\theta \\
y &= \rho\sin\theta
\end{align}

$u_1$ ifadesini silindirik koordinatlarda yazalım:
\begin{align}
u_1 &= 2xy^2 + x^2 \\
&= 2(\rho\cos\theta)(\rho\sin\theta)^2 + (\rho\cos\theta)^2 \\
&= 2\rho^3\cos\theta\sin^2\theta + \rho^2\cos^2\theta \\
&= \rho^2(2\rho\cos\theta\sin^2\theta + \cos^2\theta)
\end{align}

\subsection{$u_2 = xy + z$}

Benzer şekilde:
\begin{align}
u_2 &= xy + z \\
&= (\rho\cos\theta)(\rho\sin\theta) + z \\
&= \rho^2\cos\theta\sin\theta + z
\end{align}

\subsection{$u_3 = x^2y + z^2y + z$}

Ve son olarak:
\begin{align}
u_3 &= x^2y + z^2y + z \\
&= (\rho\cos\theta)^2(\rho\sin\theta) + z^2(\rho\sin\theta) + z \\
&= \rho^3\cos^2\theta\sin\theta + \rho z^2\sin\theta + z
\end{align}

\section{Ödev: Küresel Koordinatlara Uygulama}

Soru 3'teki işlemleri küresel koordinatlara uygulayalım.

\subsection{Küresel Koordinatlarda Birim Vektörler}

Küresel koordinatlarda birim vektörleri kartezyen birim vektörler cinsinden ifade edelim:

\begin{align}
\vec{e}_r &= \sin\phi\cos\theta\, \vec{e}_x + \sin\phi\sin\theta\, \vec{e}_y + \cos\phi\, \vec{e}_z \\
\vec{e}_\theta &= -\sin\theta\, \vec{e}_x + \cos\theta\, \vec{e}_y \\
\vec{e}_\phi &= \cos\phi\cos\theta\, \vec{e}_x + \cos\phi\sin\theta\, \vec{e}_y - \sin\phi\, \vec{e}_z
\end{align}

Burada $r$ radyal mesafe, $\theta$ azimut açısı (xy-düzlemindeki açı) ve $\phi$ polar açıdır (z-ekseni ile yapılan açı).

\subsection{Koordinat Dönüşümleri}

Küresel koordinatlar ile kartezyen koordinatlar arasındaki ilişki:

\begin{align}
x &= r \sin\phi \cos\theta \\
y &= r \sin\phi \sin\theta \\
z &= r \cos\phi
\end{align}

\subsection{Dönüşüm Matrisi}

Küresel birim vektörleri kartezyen birim vektörler cinsinden ifade eden dönüşüm matrisi:

\begin{align}
\begin{bmatrix} \vec{e}_r \\ \vec{e}_\theta \\ \vec{e}_\phi \end{bmatrix} = 
\begin{bmatrix} 
\sin\phi\cos\theta & \sin\phi\sin\theta & \cos\phi \\
-\sin\theta & \cos\theta & 0 \\
\cos\phi\cos\theta & \cos\phi\sin\theta & -\sin\phi
\end{bmatrix}
\begin{bmatrix} \vec{e}_x \\ \vec{e}_y \\ \vec{e}_z \end{bmatrix}
\end{align}

\subsection{Konum Vektörünün İfadesi}

Küresel koordinatlarda konum vektörü:
\begin{align}
\vec{r} = r\vec{e}_r
\end{align}

Bu ifade, küresel koordinatlarda konum vektörünün sadece radyal bileşene sahip olduğunu gösterir.

\subsection{Skaler Alanların Dönüşümü}

Aynı skaler alanları küresel koordinatlarda ifade edelim:

\subsubsection{$u_1 = 2xy^2 + x^2$}

Küresel koordinatlarda:
\begin{align}
u_1 &= 2xy^2 + x^2 \\
&= 2(r\sin\phi\cos\theta)(r\sin\phi\sin\theta)^2 + (r\sin\phi\cos\theta)^2 \\
&= 2r^3\sin^3\phi\cos\theta\sin^2\theta + r^2\sin^2\phi\cos^2\theta \\
&= r^2\sin^2\phi(2r\sin\phi\cos\theta\sin^2\theta + \cos^2\theta)
\end{align}

\subsubsection{$u_2 = xy + z$}

Küresel koordinatlarda:
\begin{align}
u_2 &= xy + z \\
&= (r\sin\phi\cos\theta)(r\sin\phi\sin\theta) + r\cos\phi \\
&= r^2\sin^2\phi\cos\theta\sin\theta + r\cos\phi
\end{align}

\subsubsection{$u_3 = x^2y + z^2y + z$}

Küresel koordinatlarda:
\begin{align}
u_3 &= x^2y + z^2y + z \\
&= (r\sin\phi\cos\theta)^2(r\sin\phi\sin\theta) + (r\cos\phi)^2(r\sin\phi\sin\theta) + r\cos\phi \\
&= r^3\sin^3\phi\cos^2\theta\sin\theta + r^3\cos^2\phi\sin\phi\sin\theta + r\cos\phi
\end{align}

\end{document}